%-------------------------------------------------------------------------------
%                            	BAB V
%               		KESIMPULAN DAN SARAN
%-------------------------------------------------------------------------------

\chapter{KESIMPULAN DAN SARAN}

\section{Kesimpulan}
	\begin{enumerate}
		\item
		Kondisi jaringan wireless UGM-Hotspot pada kondisi awal memiliki coverage yang baik (diatas -70dbm) rata-rata setiap lantai sebesar 45,30\% dari keseluruhan luas gedung.
		\item
		Penambahan access point dapat menambah coverage sinyal baik (diatas - 70dbm) dari jaringan wireless UGM-Hotspot di gedung JTETI dengan rata-rata setiap lantai sebesar 26,33\% sehingga area coverage sinyal baik rata-rata setelah penambahan access point baru menjadi 71,63\%.
		\item
		Semakin dekat titik dengan access point throughput yang didapatkan akan semakin besar, namun throughput tetap dapat berubah bergantung jumlah user yang ada.
		\item
		Maksimal throughput yang didapat pada sinyal lemah hanya mencapai 64,81\% dari maksimal bandwidth perangkat wireless sediakan.
		\item
		Throughput pada daerah yang dilakukan penelitian seluruhnya mengalami penambahan nilai throughput hingga mendekati maksimal bandwidth yaitu 92,59\% setelah dilakukan penambahan access point.
	\end{enumerate}


\section{Saran}
	\begin{enumerate}
		\item
		Kondisi jaringan wireless UGM-Hotspot pada kondisi awal memiliki coverage yang baik (diatas -70dbm) rata-rata setiap lantai sebesar 45,30\% dari keseluruhan luas gedung.
		\item
		Penambahan access point dapat menambah coverage sinyal baik (diatas - 70dbm) dari jaringan wireless UGM-Hotspot di gedung JTETI dengan rata-rata setiap lantai sebesar 26,33\% sehingga area coverage sinyal baik rata-rata setelah penambahan access point baru menjadi 71,63\%.
		\item
		Semakin dekat titik dengan access point throughput yang didapatkan akan semakin besar, namun throughput tetap dapat berubah bergantung jumlah user yang ada.
		\item
		Maksimal throughput yang didapat pada sinyal lemah hanya mencapai 64,81\% dari maksimal bandwidth perangkat wireless sediakan.
		\item
		Throughput pada daerah yang dilakukan penelitian seluruhnya mengalami penambahan nilai throughput hingga mendekati maksimal bandwidth yaitu 92,59\% setelah dilakukan penambahan access point.
	\end{enumerate}

	