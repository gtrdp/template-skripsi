%Template pembuatan proposal skripsi.
\documentclass{jtetiskripsi}

%Untuk prefiks pada daftar gambar dan tabel
\usepackage[titles]{tocloft}
\renewcommand\cftfigpresnum{Gambar\  }
\renewcommand\cfttabpresnum{Tabel\   }

%Untuk hyperlink dan table of content
\usepackage{hyperref}

\newlength{\mylenf}
\settowidth{\mylenf}{\cftfigpresnum}
\setlength{\cftfignumwidth}{\dimexpr\mylenf+1.5em}
\setlength{\cfttabnumwidth}{\dimexpr\mylenf+1.5em}

%Untuk Bold Face pada Keterangan Gambar
\usepackage[labelfont=bf]{caption}

%-----------------------------------------------------------------
%Disini awal masukan untuk data proposal skripsi
%-----------------------------------------------------------------
\titleind{PENGEMBANGAN \emph{GATEWAY} BERBASIS \emph{EMBEDDED DEVICE} UNTUK INTEROPERABILITAS JARINGAN SENSOR NIRKABEL DAN PROTOKOL INTERNET}

\fullname{GUNTUR DHARMA PUTRA}

\idnum{09/284593/TK/35393}

\approvaldate{3 Februari 2014}

\degree{Sarjana Teknik Elektro}

\yearsubmit{2014}

\program{Teknik Elektro}

\dept{Teknik Elektro dan Teknologi Informasi}

\firstsupervisor{Sigit Basuki Wibowo, S.T., M.Eng.}
\firstnip{1976 0501 2002 12 1 002}

\secondsupervisor{Bimo Sunarfri Hantono, S.T., M.Eng.}
\secondnip{1977 0131 2002 12 1 003}

%-----------------------------------------------------------------
%Disini akhir masukan untuk data proposal skripsi
%-----------------------------------------------------------------

\begin{document}

\cover

\approvalpage

%-----------------------------------------------------------------
%Disini awal masukan Acknowledment
%-----------------------------------------------------------------
\acknowledgment
\begin{flushright}
\emph{Karya sederhana ini kupersembahkan \\
buat Bapak, Ibu,\\dan Adik tercinta}
\end{flushright}

%-----------------------------------------------------------------
%Disini awal masukan untuk Prakata
%-----------------------------------------------------------------
\preface
Segala puji dan syukur semata-mata hanya untuk Allah SWT, karena atas segala
rahmat, hidayah dan bantuan-Nya jualah maka akhirnya Tesis dengan judul
Analisis Teoretis Pemantulan dan Pembiasan Gelombang Elektromagnet Pada
Bahan Magnetik Non Linear Orde Dua ini telah selesai penulis susun.

Telah banyak bantuan yang penulis peroleh selama dalam penulisan Tesis ini
, untuk itu tak lupa penulis ucapkan terima kasih yang sebesar-besarnya
kepada:
\begin{enumerate}
\item{Bapak Sarjiya, S.T., M.T., Ph.D., selaku Ketua Jurusan Teknik Elektro dan Teknologi Informasi Fakultas Teknik Universitas Gadjah Mada.}
\item{Bapak...selaku dosen pembimbing pertama ...}
\item{Ibu... selaku dosen pembimbing kedua ...}
\item{Bapak... selaku dosen pembimbing akademis.}
\item{Bapak dan Mama yang selama ini telah sabar membimbing dan mendoakan
penulis tanpa kenal untuk selama-lamanya,}
\item{Prof. Drs. H. Muslim, Ph. D, selaku Pembimbing Utama, yang telah
memberikan ilmunya kepada penulis serta dengan penuh kesabaran membimbing penulis,}
\item{Drs. Kamsul Abraha, Ph. D, selaku Pembimbing Pendamping yang telah
memberikan inspirasi kepada penulis,} 
\item{Dr. Pekik Nurwantoro dan Dr. rer. nat. M. Farchani Rasyid
yang telah memperkenalkan sistem operasi LINUX dan \LaTeX{} kepada penulis serta
memberikan bimbingan penggunaan \LaTeX{} tersebut dengan sabar,} 
\item{Segenap staf dan karyawan di jurusan Fisika FMIPA UGM, yang telah
banyak bekerjasama dengan penulis selama belajar di FMIPA UGM,} 
\item{Sahabat saya M. Rizal Ginanjar, yang selalu bersedia membantu penulis ketika
menyelesaikan masalah-masalah komputer.} 
\end{enumerate}

Tesis ini tentunya tidak lepas dari segala kekurangan dan kelemahan, untuk itu
segala kritikan dan saran yang bersifat membangun guna kesempurnaan Tesis ini
sangat diharapkan. Semoga tesis ini dapat bermanfaat bagi kita semua dan lebih
khusus lagi bagi pengembagan ilmu fisika.

\begin{tabular}{p{7.5cm}c}
&Yogyakarta, 15 Januari 2014\\
&\\
&\\
&Penulis
\end{tabular}

%-----------------------------------------------------------------
%Disini akhir masukan untuk muka skripsi
%-----------------------------------------------------------------
\tableofcontents
\addcontentsline{toc}{chapter}{DAFTAR ISI}
\listoftables
\addcontentsline{toc}{chapter}{DAFTAR TABEL}
\listoffigures
\addcontentsline{toc}{chapter}{DAFTAR GAMBAR}

%-----------------------------------------------------------------
%Daftar Singkatan [Optional]
%-----------------------------------------------------------------
\singkatan
\noindent
\begin{tabular}{lp{3pt}l}

\textbf{A}\\
AP & & Access Point\\
\\

\textbf{F}\\
FTDI & & Future Technology Devices International\\
FUSE & & Filesystem in Userspace\\
\\

\textbf{J}\\
JTETI & & Jurusan Teknik Elektro dan Teknologi Informasi\\
\\

\textbf{L}\\
LAN & & Local Area Network\\
\\

\textbf{R}\\
RF & & Radio Frequency\\
\\

\textbf{S}\\
SFTP & & Secure Shell File Transfer Protocol\\
SSHFS & & Secure Shell Filesystem\\
\\

\textbf{U}\\
UGM & & Universitas Gadjah Mada\\
\\

\textbf{W}\\
WAP & & Wireless Access Point\\
WLAN & & Wireless Local Area Network\\
WSN & & Wireless Sensor Network\\

\end{tabular}

%-----------------------------------------------------------------
%Disini awal masukan Intisari
%-----------------------------------------------------------------
\begin{abstractind}
Penggunaan \emph{Wireless Sensor Network} (WSN) untuk gedung dan perumahan semakin populer karena dapat dimanfaatkan untuk berbagai kepentingan seperti \emph{home automation} dan \emph{home surveillance}. Oleh karena itu, untuk meningkatkan fleksibilitas penggunaan WSN, diperlukan sistem pengendalian yang dapat dikendalikan secara jarak jauh. Padahal pada umumnya, WSN dikendalikan oleh sebuah pengendali utama berada di sekitar tempat WSN itu berada.

Penelitian ini mengusulkan integrasi dari WSN dengan \emph{Internet Protocol} (IP) yang memungkinkan WSN dapat dikendalikan dimanapun dan dengan apapun asalkan masih terhubung dengan jaringan internet. Penelitian ini memanfaatkan infrastruktur jaringan data yang sangat populer dan terhubung ke internet, yaitu jaringan area lokal nirkabel atau dikenal dengan nama WiFi. Salah satu perangkat utama dalam jaringan WiFi adalah \emph{Access Point} (AP) yang berfungsi sebagai koordinator simpul. Selain itu, AP juga berfungsi sebagai gateway yang menghubungkan berbagai piranti yang terhubung padanya ke internet. Oleh karena itu, penelitian ini akan mengembangkan perangkat lunak yang akan ditanamkan ke dalam AP sehingga menjadikan AP mempunyai kemampuan sebagai gateway untuk kedua jaringan WiFi dan beberapa protokol WSN ke dalam jaringan internet.


\bigskip
\noindent
\textbf{Kata kunci :} \emph{wireless sensor network}, \emph{Internet Protocol}, WiFi, interoperabilitas.
\end{abstractind}

\begin{abstracteng}
\emph{
Wireless Sensor Network (WSN) usage for buildings and household has been increasingly popular because it offers many benefits, such as home automation and home surveilliance. Therefore, in order to increase WSN flexibility usage, remote controlling which enables administration is needed. In fact, generally WSN is controlled by a coordinator (sink node) which is located near the WSN area itself.}

\emph{
This research proposes integration of WSN and Internet Protocol (IP), that enables remote controlling and administration through the Internet Protocol stack. This research utilizes the wireless local area network or commonly known as WiFi. One of main components on this WiFi network is Access Point (AP) that acts as node coordinator. Furthermore, AP also serves as a gateway that connects multiple devices, that is connected to the AP, to the Internet. Therefore, this research will develop a software which is going to be implemented to the AP so that the AP has a capability as a gateway for both WiFi network and several WSN protocols to the Internet.}


\bigskip
\noindent
\textbf{\emph{Keywords :}} \emph{wireless sensor network, Internet Protokol, WiFi, interoperability}.
\end{abstracteng}
%-----------------------------------------------------------------
%Disini akhir masukan Intisari
%-----------------------------------------------------------------

%-----------------------------------------------------------------
%Disini awal masukan untuk Bab
%-----------------------------------------------------------------
%-------------------------------------------------------------------------------
% 								BAB I
% 							LATAR BELAKANG
%-------------------------------------------------------------------------------

\chapter{LATAR BELAKANG}

\section{Latar Belakang Masalah}
Jaringan sensor nirkabel, atau yang juga dikenal dengan istilah \emph{Wireless Sensor Network} (WSN), adalah jaringan simpul sensor otonom terdistribusi yang dapat berkomunikasi satu sama lain secara nirkabel. WSN tidak hanya digunakan dalam lingkup penelitian tapi juga sudah digunakan serara luas dalam implementasinya di pengamatan lingkungan, fasilitas rumah cerdas, pelacakan alam dan ekologis, pengendalian kualitas dalam teknik industri dan pembangunan, prediksi tanah longsor, dan lain-lain.

Penggunaan WSN untuk sebuah gedung dan perumahan kian populer karena dapat dimanfaatkan untuk berbagai kepentingan. Contoh penerapan WSN dalam rumah yang sangat populer adalah \emph{home automation} yaitu proses automatisasi segala rutinitas yang ada di rumah. Sebagai contoh, sang pemilik rumah harus menyalakan lampu di kala waktu sudah senja dan atau menyalakan pendingin ruangan saat pemilik baru saja pulang dari bekerja. Segala sesuatu yang mungkin untuk diautomatisasi, dapat terealisasi dengan bantuan WSN. Contoh lain penerapan WSN dalam rumah adalah \emph{home surveillance} yaitu pemanfaatan WSN untuk mengawasi tiap sudut rumah secara \emph{realtime}. Dengan ini, sang pemilik rumah tidak perlu lagi khawatir jika rumahnya kurang pengawasan karena mengawasi rumah menjadi semakin mudah dengan bantuan teknologi WSN ini.

Pada umumnya, WSN dikendalikan oleh sebuah \emph{sink node} yang berada dekat pada wilayah jaringan sensornya. Padahal, pengendalian WSN sebagai fasilitas rumah cerdas acap kali memerlukan pendendalian secara jarak jauh karena sang pemilik rumah tidak selalu berada di dalam rumah.

Selain itu, sebuah \emph{sink node}-pun juga memiliki keterbatasan, terlebih dalam hal menangani beberapa WSN yang berasal dari berbagai \emph{vendor}. Sebagai contoh, sebuah \emph{sink node} keluaran IQRF tidak dapat menangani WSN lain selain dari IQRF. Begitu pula \emph{sink node} produk XBee, misalnya, tidak bisa menangani sama sekali sensor-sensor selain produk XBee.

Lebih jauh lagi, protokol yang digunakan dalam WSN adalah protokol yang sifatnya \emph{proprietary} yang sifatnya tertutup dan tidak mendukung interoperabilitas. Hal ini menyebabkan tiap WSN dari \emph{vendor} yang berbeda tidak dapat saling berkomunikasi.

Protokol Internet menawarkan fleksibilitas dan interoperabilitas WSN dengan peranti lain, bahkan dengan peranti selain WSN seperti komputer dan telepon seluler. Namun demikian, protokol Internet boros dalam pengkonsumsian daya listrik. Ada beberapa cara untuk mengintegrasikan IP dengan WSN, salah satu caranya adalah dengan membangun \emph{gateway} yang terhubung dengan \emph{sink node} dari WSN.


\section{Rumusan Masalah}
Bagaimana cara membangun sistem pengendalian WSN yang dapat dikendalikan secara jarak jauh dengan interoperabilitas WSN dan protokol Internet. Selain itu, sistem pengendalian yang dibangun juga harus dapat mengintegrasikan beberapa vendor WSN ke dalam sebuah \emph{gateway} yang sama.


\section{Batasan Masalah}
Batasan masalah pada penelitian ini adalah:
\begin{enumerate}
\item Penelitian ini difokuskan pada interoperabilitas beberapa \emph{vendor} WSN dan protokol Internet.
\item Tipe WSN yang digunakan dalam penelitian ini dibatasi dua buah.
\item Pengujian yang dilakukan hanya sebatas eksperimen dalam lingkup laboratorium.
\item Purwarupa yang dihasilkan akan diimplementasikan pada sebuah \emph{Access Point} (AP).
\end{enumerate}


\section{Tujuan Penelitian}
Tujuan penelitian ini adalah mempelajari kemungkinan pengembangan perangkat lunak yang akan ditanamkan ke dalam sebuah AP untuk difungsikan sebagai gateway sehingga mampu digunakan untuk mengintegrasikan jaringan WiFi dan beberapa protokol WSN ke jaringan internet dengan biaya yang murah.


\section{Manfaat Penelitian}
Dengan terhubungnya WSN ke jaringan internet dimungkinkan pengembangan aplikasi WSN yang dapat diakses melalui jaringan internet. Terhubungnya WSN ke jaringan internet akan membuka kemungkinan pengembangan layanan-layanan yang lebih beragam terutama layanan yang berbasis IP. Hal ini sejalan dengan perkembangan teknologi komunikasi yang menuju konvergensi penggunaan IP.

Selain itu, pengintegrasian gateway untuk WiFi dan WSN dalam satu piranti juga membuka peluang besar untuk memecahkan persoalan interoperabilitas perangkat keras dari berbagai \emph{vendor} dan kemudahan sistem.


\section{Keaslian Penelitian}
Penelitian ini tidak untuk menguji hipotesis baru melainkan merupakan pengembangan perangkat lunak yang akan ditanamkan ke dalam gateway sehingga mampu menghubungkan jaringan WiFi dan WSN ke jaringan internet. Penelitian ini akan meningkatkan fungsi AP menjadi gateway yang menghubungkan WiFi dan WSN dengan jaringan internet.


\section{Sistematika Penulisan}
\noindent
\textbf{BAB I : PENDAHULUAN}

Pada bab ini dijelaskan latar belakang, rumusan masalah, batasan, tujuan, manfaat, keaslian penelitian, dan sistematika penulisan.\\

\noindent
\textbf{BAB II : TINJAUAN PUSTAKA DAN LANDASAN TEORI}

Pada bab ini dijelaskan teori-teori dan penelitian terdahulu yang digunakan sebagai acuan dan dasar dalam penelitian.\\

\noindent
\textbf{BAB III : METODOLOGI PENELITIAN}

Pada bab ini dijelaskan metode yang digunakan dalam penelitian meliputi langkah kerja, pertanyaan penilitian, alat dan bahan, serta tahapan dan alur penelitian.\\

\noindent
\textbf{BAB IV : HASIL DAN PEMBAHASAN}

Pada bab ini dijelaskan hasil penelitian dan pembahasannya.\\

\noindent
\textbf{BAB V : KESIMPULAN DAN SARAN}

Pada bab ini ditulis kesimpulan akhir dari penelitian dan saran untuk pengembangan penelitian selanjutnya.\\

% Baris ini digunakan untuk membantu dalam melakukan sitasi
% Karena diapit dengan comment, maka baris ini akan diabaikan
% oleh compiler LaTeX.
\begin{comment}
\bibliography{daftar-pustaka}
\end{comment}


%!TEX root = ./template-skripsi.tex
%-------------------------------------------------------------------------------
%                            BAB II
%               TINJAUAN PUSTAKA DAN DASAR TEORI
%-------------------------------------------------------------------------------

\chapter{TINJAUAN PUSTAKA DAN DASAR TEORI}                

\section{Tinjauan Pustaka}
  Lorem ipsum is a pseudo-Latin text used in web design, typography, layout, and printing in place of English to emphasise design elements over content. It's also called placeholder (or filler) text. It's a convenient tool for mock-ups. It helps to outline the visual elements of a document or presentation, eg typography, font, or layout. Lorem ipsum is mostly a part of a Latin text by the classical author and philospher Cicero. Its words and letters have been changed by addition or removal, so to deliberately render its content nonsensical; it's not genuine, correct, or comprehensible Latin anymore. While lorem ipsum's still resembles classical Latin, it actually has no meaning whatsoever. As Cicero's text doesn't contain the letters K, W, or Z, alien to latin, these, and others are often inserted randomly to mimic the typographic appearence of European languages, as are digraphs not to be found in the original. \cite{DaSilvaCampos2011}

\section{Landasan Teori}
  \subsection{\LaTeX}
    Ne per tota mollis suscipit. Ullum labitur vim ut, ea dicit eleifend dissentias sit. Duis praesent expetenda ne sed. Sit et labitur albucius elaboraret. Ceteros efficiantur mei ad. Hendrerit vulputate democritum est at, quem veniam ne has, mea te malis ignota volumus.

    Eros reprimique vim no. Alii legendos volutpat in sed, sit enim nemore labores no. No odio decore causae has. Vim te falli libris neglegentur, eam in tempor delectus dignissim, nam hinc dictas an.

    Pro omnium incorrupte ea. Elitr eirmod ei qui, ex partem causae disputationi nec. Amet dicant no vis, eum modo omnes quaeque ad, antiopam evertitur reprehendunt pro ut. Nulla inermis est ne. Choro insolens mel ne, eos labitur nusquam eu, nec deserunt reformidans ut. His etiam copiosae principes te, sit brute atqui definiebas id.


      \begin{figure}[H]
        \centering
          \includegraphics{gambar/wsn}
          \caption{Jaringan sensor nirkabel.}
          \label{wsn}
      \end{figure}


  \subsection{Sublime Text}
    Et affert civibus has. Has ne facer accumsan argumentum, apeirian hendrerit persequeris pro ex. Suscipit vivendum sensibus mea at, vim ei hinc numquam, at dicit timeam dissentiet mel. At patrioque intellegebat sea, error argumentum dissentias sea in.

    Quo no atqui omnesque intellegat, ne nominavi argumentum quo. Eum ei purto oporteat dissentiet, soleat utamur an sit. Et assum dicam interpretaris quo. Cetero alterum ea vel, no possit alterum utroque nec. His fuisset quaestio ad. Has eu tritani incorrupte consequuntur, esse aliquip nec ne.

% Baris ini digunakan untuk membantu dalam melakukan sitasi
% Karena diapit dengan comment, maka baris ini akan diabaikan
% oleh compiler LaTeX.
\begin{comment}
\bibliography{daftar-pustaka}
\end{comment}


%-------------------------------------------------------------------------------
%                            BAB III
%               		METODOLOGI PENELITIAN
%-------------------------------------------------------------------------------

\chapter{METODOLOGI PENELITIAN}

\section{Alat dan Bahan}
	Alat dan bahan yang digunakan dalam penelitian ini adalah:

	\subsection{Perangkat Keras}
		\vspace{-0.5cm}

		\begin{enumerate}[a.]
		\begin{singlespace}
		\itemsep0em
			\item Kit pancar-rima IQRF TR-53B (3 unit),
			\item Kit pengunduh program CK-USB-04 (1 unit),
			\item Kit pengembangan DK-EVAL-03 (2 unit),
			\item Kit pengembangan CK-EVAL-04 (1 unit),
			\item \emph{XBee 802.15.4 Radios (Series 1)} (3 unit),
			\item \emph{XBee Explorer USB Board} (1 unit),
			\item \emph{2 channel Relay Shield For Arduino (With XBee/BTBee interface)} (2 unit),
			\item Arduino Uno (2 unit),
			\item TP-LINK MR3020 (1 unit),
			\item Kabel USB ke Serial Prolific (1 unit).
		\end{singlespace}
		\end{enumerate}

	\subsection{Perangkat Lunak}
		\vspace{-0.5cm}

		\begin{enumerate}[a.]
		\begin{singlespace}
		\itemsep0em
			\item Arduino for Mac OSX,
			\item CoolTerm,
			\item Driver FTDI for Mac OSX,
			\item PHP, MySQL, dan uHTTPd,
			\item Python dan pustaka PySerial,
			\item IQRF IDE v 2.08 for TR-53B,
			\item SSHFS,
			\item Sublime Text 3.
		\end{singlespace}
		\end{enumerate}

\section{Alur Penelitian}
	\subsection{Pra Penelitian}
		Sebelum penelitian dimulai, dilakukan studi literatur terkait dengan sistem yang akan dibangun. Selain itu, analisis kebutuhan juga dirancang pada tahap ini. Setelah semua selesai, dilajutkan dengan penulisan proposal penelitian.

	\subsection{Pengembangan Aplikasi}
		Ada tiga aplikasi yang akan dibangun, yaitu aplikasi berbasis bahasa C untuk masing sensor-sensor IQRF dan Arduino Uno, aplikasi berbasis web yang nantinya akan berinteraksi langsung dengan pengguna, dan aplikasi berbasis bahasa Python untuk mengomunikasikan sensor-sensor dengan aplikasi berbasis web.

		Aplikasi untuk sensor-sensor IQRF terdiri dari dua bagian, yaitu aplikasi untuk sensor koordinator dan sensor simpul. Namun demikian, aplikasi sama-sama ditulis dan kembangkan menggunakan Sublime Text 3. Setelah kode sesumber untuk aplikasi selesai dibuat, kode sesumber dikompiliasi menggunakan IDE IQRF untuk kemudian diunggah ke sensor menggunakan bantuan aplikasi yang sama. Aplikasi yang dikembangkan adalah hasil fork dari iHome, aplikasi rumah hijau yang dikembangkan oleh Wibowo, et. al.

		Sedangkan aplikasi untuk Arduino Uno, yang bertugas menyala-matikan relay dengan komunikasi berbasis ZigBee, dikembangkan dengan aplikasi Arduino for Mac OSX dengan bahasa pemrograman C. Proses kompilasi dan pengunggahan dilakukan dengan bantuan aplikasi yang sama.

		

	\subsection{Evaluasi dan Perbaikan}
		Evaluasi dilakukan dengan pengujian apakah sistem sudah berlajan dengan semestinya. Kemudian Perbaikan dilakukan dengan bantuan SSHFS agar AP dapat mengakses direktori yang tersimpan pada komputer karena \emph{coding} tidak dilakukan pada AP itu sendiri, melainkan komputer.

	\subsection{Pasca Penelitian}

	\subsection{Diagram Alir Penelitian}


\section{Tahapan Pelaksanaan}
	Rancangan arsitektur yang akan digunakan pada penelitian ini diilustrasikan seperti pada Gambar \ref{wifi}. Pada gambar tersebut diilustrasikan sebuah sistem yang terdiri atas dua buah WSN dengan protokol yang berbeda dan satu buah jaringan nirkabel lokal (WiFi). Protokol WSN yang akan digunakan dalam penelitian ini adalah dari IQRF dan ZigBee. Pelaksanaan penelitian ini akan dibagi menjadi tiga paket pekerjaan (Work Package, WP).

	\noindent\textbf{WP 1: Perancangan Perangkat Lunak}

	Pada tahap ini akan dilakukan studi literatur yang dititikberatkan pada sistem operasi (Operating System, OS) untuk piranti tertanam (embedded device). Langkah selanjutnya adalah rerancangan perangkat lunak yang akan ditanamkan pada Access Point (AP). Perangkat lunak yang akan ditanamkan harus bekerja secara efisien karena kemampuan komputasi yang terbatas pada AP.

		\begin{figure}[ht!]
		  \centering
		    \includegraphics{gambar/wifi}
		    \caption{Arsitektur WSN dan WiFi dengan sebuah AP.}
		    \label{wifi}
		\end{figure}

	\noindent\textbf{WP 2: Implementasi Perangkat Lunak}

	Implementasi perangkat lunak dilakukan pada tahap ini. Langkah pertama yang dilakukan adalah memastikan bahwa WSN dapat terhubungan dengan internet sesuai dengan yang direncanakan. Langkah selanjutnya adalah memastikan bahwa jaringan WiFi tidak mengalami gangguan setelah perangkat lunak yang baru tertanam pada AP. Penambahan layanan-layanan yang diperlukan dapat pula dilakukan pada tahap ini.

	\noindent\textbf{WP 3: Integrasi dan Pengujian Seluruh Sistem}

	Jika jaringan WiFi dan dua protokol WSN masing-masing dapat berhubungan dengan internet, maka pada tahap ini akan dilakukan pengujian sistem secara keseluruhan. Pengujian dinaikkan dari skala lab menjadi skala \emph{test-bed}. Pengujian dalam \emph{test-bed} dilakukan untuk menjamin bahwa sistem yang dikembangkan bekerja sesuai dengan yang direncanakan.


\section{Jadwal Kegiatan}
	Penelitian direncanakan akan dilaksanakan selama enam bulan. Rincian rencana jadwal penelitian dicantumkan dalam tabel berikut.

	\begin{center}
	Tabel 3.1. Jadwal Penelitian.
	\end{center}
	\vspace{-0.5cm}
	\begin{figure}[ht!]
	  \centering
	    \includegraphics[width=13cm]{gambar/timeline}
	\end{figure}

%-------------------------------------------------------------------------------
%                            BAB IV
%               		HASIL DAN PEMBAHASAN
%-------------------------------------------------------------------------------

\chapter{HASIL DAN PEMBAHASAN}
	\section{Analisis Kebutuhan Sistem}
		\subsection{Fitur-Fitur Aplikasi}
		\subsection{\emph{Use Case Diagram}}
		\subsection{Diagram Arsitektur Sistem}
		\subsection{SDLC}
	
	\section{Perancangan Aplikasi}
		\subsection{Peta Situs}
		\subsection{Diagram Alir Aplikasi}
		\subsection{Persiapan Pra Pengembangan Aplikasi}
			\begin{enumerate}
				\item Konfigurasi Router
			\end{enumerate}
		\subsection{Pengembangan Aplikasi WSN}
		\subsection{Pengembangan Aplikasi Python}
		\subsection{Pengembangan Aplikasi Berbasis Web}
		\subsection{Evaluasi dan Perbaikan}
		\subsection{\emph{Screenshot} Aplikasi}

	\section{Analisis Unjuk Kerja Aplikasi}
		\subsection{Hasil Uji Coba Aplikasi}
		\subsection{Masalah dan Penyelesaian}


%!TEX root = ./template-skripsi.tex
%-------------------------------------------------------------------------------
%                            	BAB V
%               		KESIMPULAN DAN SARAN
%-------------------------------------------------------------------------------

\chapter{KESIMPULAN DAN SARAN}

\section{Kesimpulan}
	Berdasarkan hasil analisis dan pengujian fungsional aplikasi ini, didapat kesimpulan sebagai berikut:

	\begin{enumerate}
		\item Lorem ipsum is a pseudo-Latin text used in web design, typography, layout, and printing in place of English to emphasise design elements over content. 
		
		\item It's also called placeholder (or filler) text. It's a convenient tool for mock-ups. 
		
		\item It helps to outline the visual elements of a document or presentation, eg typography, font, or layout. Lorem ipsum is mostly a part of a Latin text by the classical author and philospher Cicero.

		\item Its words and letters have been changed by addition or removal, so to deliberately render its content nonsensical; it's not genuine, correct, or comprehensible Latin anymore. 
	\end{enumerate}


\section{Saran}
	\begin{enumerate}
		\item Lorem ipsum is a pseudo-Latin text used in web design, typography, layout, and printing in place of English to emphasise design elements over content. 
		
		\item It's also called placeholder (or filler) text. It's a convenient tool for mock-ups. 
		
		\item It helps to outline the visual elements of a document or presentation, eg typography, font, or layout. Lorem ipsum is mostly a part of a Latin text by the classical author and philospher Cicero.

		\item Its words and letters have been changed by addition or removal, so to deliberately render its content nonsensical; it's not genuine, correct, or comprehensible Latin anymore. 
	\end{enumerate}

	
% Baris ini digunakan untuk membantu dalam melakukan sitasi
% Karena diapit dengan comment, maka baris ini akan diabaikan
% oleh compiler LaTeX.
\begin{comment}
\bibliography{daftar-pustaka}
\end{comment}


%-----------------------------------------------------------------
%Disini akhir masukan Bab
%-----------------------------------------------------------------

%-----------------------------------------------------------------
%Disini awal masukan untuk Daftar Pustaka
%-----------------------------------------------------------------
%%\nocite{Abel2010,Guerbas201350}
%%\bibliography{research-plan}
%%\bibliographystyle{plainnat}
\begin{thebibliography}{9}

\bibitem[satu(2013)]{satu01}
Spinar, R., dkk, ``Demo Abstract: Efficient Building Management with IP- based Wireless Sensor Network'', , 6th European Conference on Wireless Sensor Networks. Cork, Ireland 11-13 February 2009.

\bibitem[dua(2013)]{dua02}
Adam Dunkels, Thiemo Voigt, Niclas Bergman, dan Mats Jonsson ``The Design and Implementation of an IP-based Sensor Network for Intrusion Monitoring'', Swedish National Computer Networking Workshop, Sweden, 2004.

\bibitem[tiga(2013)]{tiga03}
Sigit B. Wibowo, dan Widyawan, ``Wireless Sensor Network and Internet Protocol Integration with COTS'', 2013 AUN/SEED-Net Regional Conference in Electrical and Electronics Engineering, Bangkok, Thailand, 2013.

\bibitem[empat(2013)]{empat04}
Dokumen online, http://www.iqrf.org/, IQRF, diakses pada Maret 2013

\bibitem[lima(2013)]{lima05}
Widyawan, Sigit B. Wibowo, dkk, ``iHome: Low-Cost Domotic for Residential Houses'', 5th AUN/SEED-Net Regional Conference on Information and Communications Technology (RCICT), Manila, Filipina, 2012.

\bibitem[enam(2013)]{enam06}
Dokumen online,https://openwrt.org/, diakses pada Maret 2013

\bibitem[tujuh(2013)]{tujuh07}
Dokumen online, http://www.digi.com/technology/rf-articles/wireless-zigbe,
diakses pada Maret 2013.

\end{thebibliography}
\addcontentsline{toc}{chapter}{DAFTAR PUSTAKA}
%-----------------------------------------------------------------
%Disini akhir masukan Daftar Pustaka
%-----------------------------------------------------------------

\end{document}