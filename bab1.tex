\chapter{LATAR BELAKANG}

\section{Latar Belakang Masalah}
Jaringan sensor nirkabel (\emph{Wireless Sensor Network}, WSN) adalah jaringan simpul (\emph{node}) sensor otonom terdistribusi yang digunakan untuk memonitor kondisi fisik atau lingkungan misalnya suhu, suara, getaran, kelembaban, dan lain-lain. Selain itu, tidak menutup kemungkinan untuk menambahkan fungsi tambahan pada setiap simpul misalnya port masukan/keluaran yang dapat digunakan sebagai pengendali aktuator yang terhubung ke piranti elektrik atau elektronis.

Penggunaan WSN untuk sebuah gedung dan rumah semakin populer karena dapat dimanfaatkan untuk berbagai kepentingan. Contoh penerapan WSN dalam rumah yang sangat populer adalah \emph{home surveillance} yaitu pemanfaatan WSN untuk mengawasi tiap sudut rumah secara \emph{realtime}. Dengan ini, sang pemilik rumah tidak perlu lagi khawatir karena rumahnya kurang pengawasan karena mengawasi rumah menjadi semakin mudah dengan bantuan WSN ini. Contoh lainnya adalah \emph{home automation} yaitu proses automatisasi segala urusan yang ada di rumah. Sebagai contoh, sang pemilik rumah harus menyalakan lampu di kala waktu sudah senja dan atau menyalakan pendingin ruangan saat pemilik baru saja pulang dari bekerja. Segala sesuatu yang mungkin untuk diautomatisasi, dapat terealisasi dengan bantuan WSN.

Pada umumnya, WSN dikendalikan oleh \emph{sink node} yang berada dekat pada wilayah jaringan sensornya. Sehingga permasalahan pada WSN adalah jika diinginkan pusat kendali berada pada tempat yang jauh dari jaringan sensornya. Solusi yang mungkin dari permasalahan ini adalah penggunaan \emph{Internet Protocol} (IP) karena jaringan IP sangat luas dan dapat diakses dimanapun dan dengan apapun.

Namun demikian, pada umumnya jaringan WSN tidak menggunakan IP, melainkan protokolnya sendiri, seperti protokol \emph{zigbee}. Oleh karena itu, diperlukan sebuah gateway yang mampu menghubungkan WSN dari berbagai macam \emph{vendor} dengan jaringan internet.


\section{Rumusan Masalah}


\section{Batasan Masalah}



\section{Tujuan Penelitian}
Tujuan penelitian ini adalah mempelajari kemungkinan pengembangan perangkat lunak yang akan ditanamkan ke dalam sebuah \emph{access point} untuk difungsikan sebagai gateway sehingga mampu digunakan untuk mengintegrasikan jaringan WiFi dan beberapa protokol WSN ke jaringan internet.


\section{Metodologi Penelitian}


\section{Manfaat Penelitian}
Dengan terhubungnya WSN ke jaringan internet dimungkinkan pengembangan aplikasi WSN yang dapat diakses melalui jaringan internet. Terhubungnya WSN ke jaringan internet akan membuka kemungkinan pengembangan layanan-layanan yang lebih beragam terutama layanan yang berbasis IP. Hal ini sejalan dengan perkembangan teknologi komunikasi yang menuju konvergensi penggunaan IP.

Selain itu, pengintegrasian gateway untuk WiFi dan WSN dalam satu piranti juga membuka peluang besar untuk memecahkan persoalan interoperabilitas perangkat keras dan kemudahan sistem.

\section{Keaslian Penelitian}


\section{Sistematika Penulisan}

